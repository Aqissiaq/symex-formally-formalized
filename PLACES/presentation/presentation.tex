
%%%%%%%%%%%%%%%%%%%%%%%%%%%%%%%%%%%%%%%%%
% Beamer Presentation
% LaTeX Template
% Version 1.0 (10/11/12)
%
% This template has been downloaded from:
% http://www.LaTeXTemplates.com
%
% License:
% CC BY-NC-SA 3.0 (http://creativecommons.org/licenses/by-nc-sa/3.0/)
%
%%%%%%%%%%%%%%%%%%%%%%%%%%%%%%%%%%%%%%%%%

%----------------------------------------------------------------------------------------
%   PACKAGES AND THEMES
%----------------------------------------------------------------------------------------

\documentclass{beamer}

\usepackage{tikz}
\usetikzlibrary{cd}

\mode<presentation> {

% The Beamer class comes with a number of default slide themes
% which change the colors and layouts of slides. Below this is a list
% of all the themes, uncomment each in turn to see what they look like.

%\usetheme{default}
%\usetheme{AnnArbor}
%\usetheme{Antibes}
%\usetheme{Bergen}
%\usetheme{Berkeley}
%\usetheme{Berlin}
%\usetheme{Boadilla}
%\usetheme{CambridgeUS}
%\usetheme{Copenhagen}
%\usetheme{Darmstadt}
%\usetheme{Dresden}
% \usetheme{Frankfurt}
%\usetheme{Goettingen}
%\usetheme{Hannover}
%\usetheme{Ilmenau}
%\usetheme{JuanLesPins}
%\usetheme{Luebeck}
%\usetheme{Madrid}
%\usetheme{Malmoe}
%\usetheme{Marburg}
\usetheme{Montpellier}
%\usetheme{PaloAlto}
%\usetheme{Pittsburgh}
%\usetheme{Rochester}
%\usetheme{Singapore}
%\usetheme{Szeged}
%\usetheme{Warsaw}

% As well as themes, the Beamer class has a number of color themes
% for any slide theme. Uncomment each of these in turn to see how it
% changes the colors of your current slide theme.

% \usecolortheme{albatross}
% \usecolortheme{beaver}
% \usecolortheme{beetle}
% \usecolortheme{crane}
%\usecolortheme{dolphin}
%\usecolortheme{dove}
%\usecolortheme{fly}
\usecolortheme{lily}
%\usecolortheme{orchid}
%\usecolortheme{rose}
%\usecolortheme{seagull}
%\usecolortheme{seahorse}
%\usecolortheme{whale}
%\usecolortheme{wolverine}
% \usecolortheme{spruce}

%\setbeamertemplate{footline} % To remove the footer line in all slides uncomment this line
\setbeamertemplate{footline}[page number] % To replace the footer line in all slides with a simple slide count uncomment this line

\setbeamertemplate{navigation symbols}{} % To remove the navigation symbols from the bottom of all slides uncomment this line
}

\usepackage{graphicx} % Allows including images
\graphicspath{{media/}}
% \usepackage{subfig} % Allows including images
% \usepackage{booktabs} % Allows the use of \toprule, \midrule and \bottomrule in tables
\usepackage{datetime}
\usepackage[]{bm}
\usepackage[backend=biber, sorting=none]{biblatex}
\addbibresource{../sources.bib}

\newcommand{\email}[1]{\href{mailto:#1}{\texttt{#1}}}
\renewcommand{\note}[1]{{\color{orange}#1}}
\newcommand{\faded}[2][35]{\textcolor{fg!#1}{#2}}

\DeclareMathOperator{\ifs}{\mathtt{if}}
\DeclareMathOperator{\whiles}{\mathtt{while}}
\DeclareMathOperator{\skips}{\mathtt{skip}}

\DeclareMathOperator{\fold}{\mathtt{fold}}
\DeclareMathOperator{\update}{\mathtt{update}}
\DeclareMathOperator{\true}{\mathtt{true}}
\DeclareMathOperator{\false}{\mathtt{false}}

% syntax for traces
\newcommand{\pc}[1]{\ensuremath{pc(#1)}}
\newcommand{\accC}[2][V_0]{\ensuremath{#2\Downarrow_{#1}}}
\newcommand{\accS}[2][]{\ensuremath{#2\downarrow_{#1}}}

%----------------------------------------------------------------------------------------
%   TITLE PAGE
%----------------------------------------------------------------------------------------

\title[Concurrent SE in Coq]{Concurrent Symbolic Execution with Trace Semantics in Coq}
\author[Å. A. A. Kløvstad]{Åsmund Aqissiaq Arild Kløvstad\\ \email{aaklovst@ifi.uio.no}}
\institute[UiO] {Department of Informatics\\ University of Oslo}

\newdate{PLACESdate}{22}{04}{2023}
\newdate{IFIdate}{12}{04}{2023}
\date[PLACES 2023]{PLACES 2023, \displaydate{IFIdate}} % Date, can be changed to a custom date

\titlegraphic{\includegraphics[height=3cm]{03_uio_full_logo_eng_pos}}

\begin{document}

\begin{frame}
\titlepage%
\end{frame}

\begin{frame}{Symbolic Execution}
  \begin{itemize}
    \item Execution with \emph{symbolic values}
    \item Used for: \note{cite work}
          \begin{itemize}
            \item Program analysis
            \item (Directed) testing
          \end{itemize}
    \item Challenges
          \begin{itemize}
            \item \faded{Non-termination}
            \item Concurrency and scaling
            \item Formalism
          \end{itemize}
  \end{itemize}
\end{frame}

\begin{frame}{Symbolic Execution Formally Explained~\cite{boer2021}}
  \begin{columns}[t]
    \begin{column}{.6\textwidth}
      \begin{itemize}
        \item Expressions
              \begin{itemize}
                \item with standard booleans ops
              \end{itemize}
        \item A small imperative language
        \item Substitutions $\sigma$: Var $\rightarrow$ Expr
              \begin{itemize}
                \item Application $e\sigma$: Expr
                \item identity $id : x \mapsto x$
              \end{itemize}
        \item Valuations $V$: Var $\rightarrow$ Val
              \begin{itemize}
                \item Evaluation $V(e)$: Val
              \end{itemize}
        \item $V \circ \sigma$: Valuation
              \begin{itemize}
                \item $V \circ id = V$
              \end{itemize}
      \end{itemize}
    \end{column}%
    \begin{column}{.4\textwidth}
      \begin{align*}
        e ::&= x \\
            & \mid op(e_{1},\ldots, e_{n})
      \end{align*}
      \begin{align*}
        s ::&= x := e\\
            & \mid s~;~s\\
            & \mid \ifs~e~\{s\}~\{s\}\\
            & \mid \whiles~e~\{s\}
      \end{align*}
    \end{column}
  \end{columns}
\end{frame}

\newcommand{\Sstep}[2]{\ensuremath{#1 \rightarrow #2}}
\newcommand{\Sstar}[2]{\ensuremath{#1 \rightarrow^{*} #2}}
\newcommand{\Cstep}[3][]{\ensuremath{#2 \bm{\Rightarrow}_{#1} #3}}
\newcommand{\Cstar}[3][]{\ensuremath{#2 \bm{\Rightarrow^{*}}_{#1} #3}}

\begin{frame}{SE Formally Explained: Semantics}
  \begin{itemize}
    \item Symbolic $\Sstar{(s, \sigma, \phi)}{(s', \sigma', \phi')}$
          \begin{align*}
            &\Sstep{(x := e~;~s, \sigma, \phi)}{(s, \sigma[x := e\sigma], \phi)}\\
            &\Sstep{(\ifs~e~\{s_{1}\}\{s_{2}\}~;~s, \sigma, \phi)}{(s_{1}~;~s, \sigma, \phi \land e\sigma)}\\
            &\Sstep{(\ifs~e~\{s_{1}\}\{s_{2}\}~;~s, \sigma, \phi)}{(s_{2}~;~s, \sigma, \phi \land \neg e\sigma)}
          \end{align*}
    \item Concrete $\Cstar{(s, V)}{(s', V')}$
          \begin{align*}
            &\Cstep{(x := e~;~s, V)}{(s, V[x := V(e)])}\\
            &\Cstep{(\ifs~e~\{s_{1}\}\{s_{2}\}~;~s, V)}{(s_{1}~;~s, V)}, \text{if V (e) = true}
          \end{align*}
  \end{itemize}
\end{frame}

\begin{frame}{SE Formally Explained: Correctness \& Completeness}
  \begin{theorem}[Correctness]
    If $\Sstar{(s, id, true)}{(s', \sigma, \phi)}$ and $V(\phi)$ = true, then $\Cstar{(s, V)}{(s, V \circ \sigma)}$
  \end{theorem}
  \vfill
  \begin{theorem}[Completeness]
    If $\Cstar{(s, V_{0})}{(s', V)}$ there exists $\Sstar{(s, id, true)}{(s', \sigma, \phi)}$ such that $V = V_{0} \circ \sigma$ and $V(\phi)$ = true
  \end{theorem}
  \note{This concludes my recap of other people's work. lmao}
\end{frame}

\begin{frame}{SE Formally Formalized}
  Implementing de Boer and Bonsangue's work in Coq
  \begin{itemize}
    \item Typed expressions
    \item The id-substitution
    \item Generic context machinery
    \item Inductive transition relations
    \item Stepwise reflexive-transitive closure from Relations
    \item Proofs by induction + case analysis
  \end{itemize}

  Ask me about Coq after!
\end{frame}

\begin{frame}{Parallelism}
  \begin{columns}[t]
    \begin{column}{.6\textwidth}
      \begin{itemize}
        \item Shared memory
        \item Parallel composition
        \item Extra context
              \[C ::= \square \mid (C~;~s) \mid (C~|\!|~s) \mid (s~|\!|~C)\]
        \item Nondeterministic branching in both symbolic and concrete semantics
        \item Correctness and completeness still hold!
      \end{itemize}
    \end{column}
    \begin{column}{.4\textwidth}
      \begin{align*}
        s ::&= \faded{x := e}\\
            &\faded{\mid s~;~s}\\
            &\faded[100]{\mid s~|\!|~s}\\
            &\faded{\mid \ifs~e~\{s\}~\{s\}}\\
            &\faded{\mid \whiles~e~\{s\}}\\
            &\faded[100]{\mid \skips}
      \end{align*}
      \note{the skip is actually there earlier for context reasons}
    \end{column}
  \end{columns}
\end{frame}

\begin{frame}{Trace Semantics}
  \begin{itemize}
    \item Expressive

  Consider \texttt{x := x + 1 |\!| x := x + 2}
  \[\sigma[x := x + 1][x := x + 2] = \sigma[x := x + 3] = \sigma[x := x + 2][x := x + 1]\]
  but $(x := x + 1) \curvearrowright (x := x + 2) \ne (x := x + 2) \curvearrowright (x := x + 1)$
    \item Compositional~\cite{din2022lagc}
    \item \alert{Partial Order Reduction~\cite{boer2020sympaths}}
  \end{itemize}
\end{frame}

\begin{frame}{Traces Formally Explained}
  Adapting de Boer and Bonsangue's work to traces
  \begin{columns}
    \begin{column}{.5\textwidth}
      \begin{itemize}
        \item Symbolic events: assignments and boolean guards
        \item Concrete events: assignments
        \item Final states and pc
        \item Abstraction
      \end{itemize}
    \end{column}
    \begin{column}{.5\textwidth}
      \begin{align*}
        \tau_{S} ::&= [~] \mid \tau_{S} :: (x := e) \mid \tau_{S} :: e\\
        \tau_{C} ::&= [~] \mid \tau_{C} :: (x := e)
      \end{align*}
      \begin{align*}
        \accC[V]{\tau_{C}} &\triangleq \fold~\update~V~\tau_{C}\\
        \accS[\sigma]{\tau_{S}} &\triangleq \fold~\update~\sigma~\tau_{S}\\
        \pc{\tau_{S}} &\triangleq \fold~\land~\true~\tau_{S} \\
        &\note{\text{pc also applies sub}}
      \end{align*}
      \begin{align*}
        \tau_{S}~V\text{-abstracts}~\tau_{C} \triangleq \accC[V]{\tau_{C}} = V \circ \accS[id]{\tau_{S}}
      \end{align*}
    \end{column}
  \end{columns}
\end{frame}

\begin{frame}{Trace Semantics Formally Explained}
  \begin{itemize}
    \item Symbolic $\Sstar{(s, \tau_{S})}{(s', \tau'_{S})}$
          \begin{align*}
            &\Sstep{(x := e, \tau_{S})}{(\skips, \tau_{S} :: (x :=  e))}\\
            &\Sstep{(\ifs~e~\{s_{1}\}\{s_{2}\}, \tau_{S})}{(s_{1}, \tau_{S} :: e)}\\
          \end{align*}
    \item Concrete $\Cstar[V]{(s, \tau_{C})}{(s', \tau'_{C})}$
          \begin{align*}
            &\Cstep[V]{(x := e, \tau_{C})}{(\skips, \tau_{C} :: (x := e))}\\
            &\Cstep[V]{(\ifs~e~\{s_{1}\}\{s_{2}\}, \tau_{C})}{(s_{1}, \tau_{C})}, \text{if } \accC[V]{\tau_C} (e) = \true
          \end{align*}
  \end{itemize}
\end{frame}

\begin{frame}{SE Formally Explained: Correctness \& Completeness}
  \begin{theorem}[Correctness]
    If $\Sstar{(s, \tau_{S}^{0})}{(s', \tau)}$, $\tau_{C}^{0}$ V-abstracts $\tau_{S}^{0}$, and $V(\pc{\tau})$ = true, there exists $\Cstar[V]{(s, \tau_{C}^{0})}{(s', \tau_{C})}$ such that $\tau_{S}$ V-abstracts $\tau_{C}$
  \end{theorem}
  \vfill
  \begin{theorem}[Completeness]
    If $\Cstar[V]{(s, \tau_{C}^{0})}{(s', \tau_{C})}$ there exists $\Sstar{(s, \tau_{S}^{0})}{(s', \tau_{S})}$ such that:
    \begin{itemize}
      \item $V(\pc{\tau_S})$ = true
      \item $\tau_S^0$~V-abstracts~$\tau_C^0$
      \item $\tau_S$~V-abstracts~$\tau_C$
    \end{itemize}
  \end{theorem}
  \note{Could formulate as bisimulation}
\end{frame}

\begin{frame}{Summary}
  \begin{itemize}
    \item Symbolic Execution Formally Explained, Formalized
    \item Extended with:
          \begin{itemize}
            \item Contexts
            \item Parallel Composition
            \item Traces Semantics
          \end{itemize}
  \end{itemize}
\end{frame}

\begin{frame}{Future}
  \begin{itemize}
    \item Generalize
          \begin{itemize}
            \item Language features
            \item Semantics / trace events
            \item Different notions of abstraction
                  \note{current one kinda collapses down to state}
          \end{itemize}
    \item \alert{Partial Order Reduction}
    \item Composition à la LAGC
  \end{itemize}
\end{frame}

\begin{frame}
  Thank you, questions?
\end{frame}

\begin{frame}{References}
  \printbibliography{}
\end{frame}

\end{document}
