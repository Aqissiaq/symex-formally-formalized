\documentclass[submission,copyright,creativecommons]{eptcs}
\providecommand{\event}{PLACES 2023} % Name of the event you are submitting to

% just for me
\usepackage[dvipsnames]{xcolor}
\newcommand{\note}[1]{\color{BurntOrange}{#1}\color{black}}

\usepackage{amsmath}

\newtheorem{theorem}{Theorem}

\usepackage{iftex}

\ifpdf
  \usepackage{underscore}         % Only needed if you use pdflatex.
  \usepackage[T1]{fontenc}        % Recommended with pdflatex
\else
  \usepackage{breakurl}           % Not needed if you use pdflatex only.
\fi

\title{Concurrent Symbolic Execution with Trace Semantics in Coq}
\author{Åsmund Aqissiaq Arild Kløvstad
  \footnote{joint work with Einar Broch Johnsen, Eduard Kamburjan and Silvia Lizeth Tapia Tarifa}
\institute{University of Oslo, Oslo, Norway}
\email{aaklovst@ifi.uio.no}
}

\newcommand{\titlerunning}{\title{}}
\newcommand{\authorrunning}{Åsmund A.A. Kløvstad}

\hypersetup{
  bookmarksnumbered,
  pdftitle    = {\titlerunning},
  pdfauthor   = {\authorrunning},
  pdfsubject  = {PLACES 2023 talk proposal},
  pdfkeywords = {symbolic execution, concurrency, coq, trace semantics, mechanization}
}

\begin{document}
\maketitle

\begin{abstract}
  Symbolic Execution (SE) is a technique for program analysis using symbolic expressions to abstract over program state,
  thereby covering many program states simultaneously. SE has been used since the mid 70's in both testing and analysis~\cite{king1976symbolic, boyer1975select},
  but its formal aspects have only recently begun to be explored~\cite{boer2021} and unified~\cite{Steinhoefel2022}.
  We present a Coq formalization of state- and trace-based SE with parallelism, as trace semantics present
  a more expressive alternative to state-based approaches.
  Additionally we consider an alternate approach based on reduction in a context.
\end{abstract}

\paragraph{Theory~\cite{boer2021}}
% move this sentence back into abstract?
% SE has applications in numerous settings~\cite{king1976symbolic, boyer1975select, Steinhoefel2022}.
As a basic setting we consider a small imperative language with assignment,
sequential composition, branching (\texttt{if}) and iteration (\texttt{while}).
\emph{Statements} in this language are denoted by $s, s', s_{0}\ldots$
Additionally we consider a set of \emph{variables} $Var$ and \emph{expressions} $Expr$ consisting of variables and
standard operations. Variables are denoted by $x,y,z\ldots$ and expressions by $e, e', e_{0}\ldots$
Both expressions and statements are assumed to be well-typed.
In this setting we define symbolic and concrete semantics to define notions of soundness and completeness.

With these definitions we can start to define symbolic semantics. The (symbolic) state is realized by a
\emph{substitution} -- a function $Var \rightarrow Expr$. A substitution is denoted by $\sigma, \sigma'\ldots$
and an update by $\sigma[x := e]$. Note that we can construct the identity substitution $id : x \mapsto x$,
and that a substitution can be (inductively) applied to an expression resulting
in a new expression. Denote such an application by $e\sigma$.
Finally, the symbolic semantics of the language are a transition system of triples $(s, \sigma, \phi)$
where $\phi$ is a boolean expression describing the path condition.
For the concrete semantics we require a set of \emph{values} $Val$. We then consider functions $V : Var \rightarrow Val$
and assume such a valuation is well-typed and can be used to evaluate an expression. Denote such a valuation by $V(e)$.
The concrete semantics are given by a transition system on pairs $(s, V)$,

To relate symbolic and concrete states, we observe that a valuation can be ``composed'' with a substitution
to obtain a new valuation. Denote such a composition by $V \circ \sigma$ and note that $(V \circ \sigma)(e) = V(e\sigma)$
for all expressions $e$.
This allows us to define and prove notions of soundness and completeness, where
$\rightarrow_{S/C}^{*}$ denotes the reflexive and transitive closure of the transition relations.
\begin{theorem}[Soundness~{\cite{boer2021}}]
  If $(s, id, true) \rightarrow_{S}^{*} (s', \sigma, \phi)$ and $V(\phi)$ = true,
  then $(s, V) \rightarrow_{C}^{*} (s', V \circ \sigma)$
\end{theorem}
Intuitively, every symbolic execution whose path condition is true in the initial valuation $V$
corresponds to a concrete execution with the same initial valuation and the final valuation is
the result of composing with the symbolic substitution.
\begin{theorem}[Completeness~{\cite{boer2021}}]
  If $(s, V) \rightarrow_{C}^{*} (s', V')$ then there exist $\sigma, \phi$ such that
  $(s, id, true) \rightarrow_{S}^{*} (s', \sigma, \phi)$, $V' = V \circ \sigma$ and $V(\phi)$ = true.
\end{theorem}
Intuitively, for every concrete execution there exists a symbolic execution whose path condition
is satisfied by the initial valuation and whose final substitution recovers the final concrete valuation.
The language can be extended with (recursive) procedure calls.
This is done by distinguishing local and global state and operating with a stack of \emph{closures} consisting
of a program fragment and the local state.
\paragraph{Coq Implementation}
We implement this basic setup in Coq. Expressions are divided into boolean and arithmetic expressions,
and both expressions and statements are inductive types. Boolean expressions may contain arithmetic as the
arguments to a less than or equal-operator.
Valuations and substitutions are functions from strings to natural numbers and expressions respectively and
expressions are evaluated by standard functions.
We implement the transition relations as inductive predicates on pairs of configurations and take their
reflexive and transitive closure by steps to the right.
Soundness and completeness are proven by induction on this relation and a case analysis of the final step.
The extension to procedure calls distinguishes local and global variables by construction, but is
otherwise straightforward.
\paragraph{Trace semantics}
Our contribution extends the work of de Boer and Bonsangue to trace semantics in a language with a parallel operator $s |\!| s'$.
Traces result in expressive~\cite{Mazurkiewicz1977} and compositional~\cite{din2022lagc} semantics.
Let symbolic and concrete traces be inductively defined by
\begin{align*}
  \tau_{S} ::= [\;] \mid \tau_{S} :: (x := e) \mid \tau_{S} :: b
  && \mbox{ and } &&
  \tau_{C} ::= [\;] \mid \tau_{C} :: (x := v)
\end{align*}
where $b$ is a boolean expression and $v$ a concrete value.
The final state of a trace can be recovered from an initial state by folding over
the trace and updating the state. Denote the result by $\tau\Downarrow_{V_{0}}$. Note that this is \emph{substitution}
in the case of symbolic traces, but a \emph{valuation} in the case of concrete traces.
The path condition of symbolic traces can also be recovered as the conjunction of all its boolean expressions.
Denote this path condition by $pc(\tau)$.
The semantics are given by a transition system on pairs $(s, \tau)$ -- with $\tau$ symbolic or concrete --
parametrized by an initial state. Let $\rightarrow_{\sigma}$ denote a symbolic transition from initial state $\sigma$,
and $\rightarrow_{V}$ a concrete transition from initial state $V$.
We also recover soundness and completeness.
\begin{theorem}[Trace Soundness]
  If $(s, [\;]) \rightarrow_{id}^{*} (s', \tau_{S})$ and $V_{0}(pc(\tau_{S}))$ = true,
  then there exists $\tau_{C}$ s.t $(s, [\;]) \rightarrow_{V_0}^{*} (s', \tau_{C})$ and
    $V_{0} \circ (\tau_{S}\Downarrow_{id}) = \tau_{C}\Downarrow_{V_{0}}$
\end{theorem}

\begin{theorem}[Trace Completeness]
  If $(s, [\;]) \rightarrow_{V_{0}}^{*} (s', \tau_{C})$ there exist $\tau_{S}$ s.t
  $(s, [\;]) \rightarrow_{id}^{*} (s', \tau_{S})$,\\
  $V_{0} \circ (\tau_{S} \Downarrow_{id}) = \tau_{C}\Downarrow_{V_{0}}$, and $V_{0}(pc(\tau_{S}))$ = true
\end{theorem}
Both of these theorems are proven for the language extended with procedure calls or parallel composition
by induction on the transition relation\footnote{In fact they both hold for appropriate initial traces (not just [])
but this formulation provides more clutter than insight}.
\paragraph{Discussion}
As an alternative to transition system semantics we implement reduction-in-context semantics~\cite{FELLEISEN1992235}.
With this approach we consider configurations of the form $(C[s], \tau)$ where $C[s]$ represent the program fragment
$s$ in the context $C$. This allows us to define a generic machinery for reductions in context and a smaller
``head reduction'' relation on $(s, \tau)$ specific to the symbolic or concrete semantics at hand.
The Coq implementation takes after work by Leroy~\cite{Leroy2020}.
Trace soundness and completeness can be stated analogously, and are proven by induction on the reduction relation.

\paragraph{Conclusion and Future Work}
We have provided an overview of de Boer and Bonsangue's formalization of SE in Coq.
Additionally we extended the formalization to include parallel composition and trace semantics,
and consider an alternative implementation of the semantics which results in more generic machinery
that can be re-used in the symbolic and concrete case.
In the future we aim to use this mechanized framework to investigate reduction techniques for symbolic
execution in concurrent settings and define useful notions of sound- and completeness for such reductions.

\bibliographystyle{eptcs}
% the .bib-file from phd proposal
\bibliography{sources}

\appendix
\section{Transition Rules}
\subsection{Transition relation for symbolic execution}
\begin{align*}
(x := e ; s, \sigma, \phi) & \rightarrow_{S} (s, \sigma[x := e\sigma], \phi) \\
(if \;e\; \{s_{1}\} \{s_{2}\} ; s, \sigma, \phi) & \rightarrow_{S} (s_{1} ; s, \sigma, \phi \land e\sigma) \\
(if \;e\; \{s_{1}\} \{s_{2}\} ; s, \sigma, \phi) & \rightarrow_{S} (s_{2} ; s, \sigma, \phi \land \neg e\sigma) \\
(while \;e\; \{s\} ; s', \sigma, \phi) & \rightarrow_{S} ( s ; while \;e\; \{s\} ; s', \sigma, \phi \land e\sigma) \\
(while \;e\; \{s\} ; s', \sigma, \phi) & \rightarrow_{S} (s', \sigma, \phi \land \neg e\sigma)
\end{align*}

\subsection{Transition relation for concrete execution}
\begin{align*}
(x := e ; s, V) & \rightarrow_{C} (s, V[x := V(e)]) \\
(if \;e\; \{s_{1}\} \{s_{2}\} ; s, V) & \rightarrow_{C} (s_{1} ; s, V) \mbox{if $V (e)$ = true} \\
(if \;e\; \{s_{1}\} \{s_{2}\} ; s, V) & \rightarrow_{C} (s_{2} ; s, V) \mbox{if $V (e)$ = false} \\
(while \;e\; \{s\} ; s', V) & \rightarrow_{C} (s ; while \;e\; \{s\} ; s', V) \mbox{if $V (e)$ = true} \\
(while \;e\; \{s\} ; s', V) & \rightarrow_{C} (s', V) \mbox{if $V (e)$ = false}
\end{align*}
\section{Reduction rules}
\subsection{Generic Context Rules}
\begin{align*}
  \frac{}{is\_context (s)} \mbox{identity context} && \frac{is\_context (C)}{is\_context (C[s] ; s')} \mbox{sequence} \\
  \frac{is\_context (C)}{is\_context (C[s] |\!| s')} \mbox{left-par}
  && \frac{is\_context (C)}{is\_context (s' |\!| C[s])} \mbox{right-par}
\end{align*}

\[\frac{(s, t) \rightarrow_{head} (s', t') \qquad is\_context (C)}{(C[s], t) \rightarrow (C[s'], t')}\]
Where $\rightarrow_{head}$ is either the symbolic or concrete head reduction relation given below:
\subsection{Head Reduction Rules}
\noindent\begin{minipage}{.5\linewidth}
\centering
Symbolic head reduction
\begin{align*}
  (x := e, \tau) & \rightarrow (skip, \tau :: (x := e)) \\
  (if \;b\; \{s_{1}\} \{s_{2}\}, \tau) & \rightarrow (s_{1}, \tau :: b) \\
  (if \;b\; \{s_{1}\} \{s_{2}\}, \tau) & \rightarrow (s_{2}, \tau :: \neg b) \\
  (skip ; s, \tau) & \rightarrow (s, \tau) \\
  (skip |\!| s, \tau) & \rightarrow (s, \tau) \\
  (s |\!| skip, \tau) & \rightarrow (s, \tau)
\end{align*}
\end{minipage}%
\begin{minipage}{.5\linewidth}
\centering
Concrete head reduction
\begin{align*}
  (x := e, \tau) & \rightarrow (skip, \tau :: (x := V(e))) \\
  (if \;b\; \{s_{1}\} \{s_{2}\}, \tau) & \rightarrow (if \; V (b) \; then \; s_{1} \; else \; s_{2}, \tau) \\
  (skip ; s, \tau) & \rightarrow (s, \tau) \\
  (skip |\!| s, \tau) & \rightarrow (s, \tau) \\
  (s |\!| skip, \tau) & \rightarrow (s, \tau)
\end{align*}
\end{minipage}
\end{document}
