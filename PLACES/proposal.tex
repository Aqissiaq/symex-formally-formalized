\documentclass[submission,copyright,creativecommons]{eptcs}
\providecommand{\event}{PLACES 2023} % Name of the event you are submitting to

% just for me
\usepackage[dvipsnames]{xcolor}
\newcommand{\note}[1]{\color{BurntOrange}{#1}\color{black}}

\usepackage{amsmath}

\newtheorem{theorem}{Theorem}

\usepackage{iftex}

\ifpdf
  \usepackage{underscore}         % Only needed if you use pdflatex.
  \usepackage[T1]{fontenc}        % Recommended with pdflatex
\else
  \usepackage{breakurl}           % Not needed if you use pdflatex only.
\fi

\title{Formalizing Concurrent Symbolic Execution with Trace Semantics in Coq}
\author{Åsmund Aqissiaq Arild Kløvstad
\institute{UiO\\ Oslo, Norway}
\institute{Department of Informatics}
\email{aaklovst@ifi.uio.no}
\and
Eduard \& and/or others?
\institute{UiO\\ Oslo, Norway}
\institute{Institute of Informatics}
}

\newcommand{\titlerunning}{Formalizing Concurrent Symbolic Execution in Coq}
\newcommand{\authorrunning}{Åsmund A.A. Kløvstad}

\hypersetup{
  bookmarksnumbered,
  pdftitle    = {\titlerunning},
  pdfauthor   = {\authorrunning},
  pdfsubject  = {PLACES 2023 talk proposal},               % Consider adding a more appropriate subject or description
  pdfkeywords = {symbolic execution, concurrency, coq, trace semantics} % Uncomment and enter keywords specific to your paper
}

\begin{document}
\maketitle

% \begin{abstract}
%   Symbolic Execution (SE) is a technique for program analysis using symbolic expressions to abstract over program state,
%   thereby covering many program states simultaneously. SE has been used since the mid 70's~\cite{king1976symbolic,boyer1975select}
%   in both testing and analysis, but its formal aspects have only recently begun to be explored~\cite{boer2021}
%   and unified~\cite{Steinhoefel2022}.
%   A challenge facing SE is ``state space explosion''. Even simple programs can result in a
%   large number of program states so that exploring them all becomes infeasible.
%   In a concurrent setting the problem becomes even worse and some reduction techniques are needed to reason
%   about all possible interleavings.
%   One way to represent these executions is by their traces which can allow for easy composition~\cite{din2022lagc}
%   and reordering of non-interfering events~\cite{Mazurkiewicz1977}.
%   We present a Coq formalization of sections 1-3 of de Boer et al.'s work, extend the language with parallelism, and
%   extend the model to symbolic and concrete trace semantics. Finally we consider a context-based presentation of the semantics
%   in the style of Felleisen~\cite{FELLEISEN1992235} and lay out a direction for future work.
% \end{abstract}

% \paragraph{Symbolic Execution Formally Explained}
% This section gives an overview of de Boer et al.'s formal account of SE
% followed by a description of our Coq implementation. It consists of symbolic and concrete semantics
% realized as transition systems and correctness and completeness results relating the two.

\subparagraph{de Boer et al. Formally Explained}
The most basic setting for SE, presented in \cite[sec. 2]{boer2021},
is a small imperative language with assignment, sequential composition, branching (\texttt{if})
and iteration (\texttt{while}). \emph{Statements} in this language are denoted by $s, s', s_{0}\ldots$
Additionally we consider a set of \emph{variables} $Var$ and \emph{expressions} $Expr$ consisting of variables and
standard operations. Variables are denoted by $x,y,z\ldots$ and expressions by $e, e', e_{0}\ldots$
Both expressions and statements are assumed to be well-typed.
With these definitions we can start to define symbolic semantics. The (symbolic) state is realized by a
\emph{substitution} -- a function $Var \rightarrow Expr$. A substitution is denoted by $\sigma, \sigma'\ldots$
and an update by $\sigma[x := e]$. Notably, since a lone variable is also an expressions, we can construct the
identity substitution $id : x \mapsto x$, and a substitution can be (inductively) applied to an expression resulting
in a new expression. Denote such an application by $e\sigma$.
Finally, the symbolic semantics of the language are a transition system of triples $(s, \sigma, \phi)$
where $\phi$ is a boolean expression describing the path condition.
%As an example, some transition rules are shown in \autoref{fig:trans-s}.
% \begin{figure}
%   \centering
%   \begin{minipage}[c]{0.4\linewidth}
%   \[(x := e ; s, \sigma, \phi) \rightarrow_{S} (s, \sigma[x := e\sigma], \phi) \]
%     \[(if \;e\; \{s_{1}\} \{s_{2}\} ; s, \sigma, \phi) \rightarrow_{S} (s_{1} ; s, \sigma, \phi \land e\sigma) \]
%     \[(if \;e\; \{s_{1}\} \{s_{2}\} ; s, \sigma, \phi) \rightarrow_{S} (s_{2} ; s, \sigma, \phi \land \neg e\sigma) \]
%     \caption{Symbolic transition steps}
%     \label{fig:trans-s}
%   \end{minipage}
%   \hfill
%   \begin{minipage}[c]{0.4\linewidth}
%     \[(x := e ; s, V) \rightarrow_{C} (s, V[x := V(e)]) \]
%     \[(if \;e\; \{s_{1}\} \{s_{2}\} ; s, V) \rightarrow_{C} (s_{1} ; s, V),\; \mbox{if V(e) = true} \]
%     \[(if \;e\; \{s_{1}\} \{s_{2}\} ; s, V) \rightarrow_{C} (s_{2} ; s, V),\; \mbox{if V(e) = false} \]
%     \caption{Concrete transition steps}
%     \label{fig:trans-c}
%   \end{minipage}
% \end{figure}
For the concrete semantics we require a set of \emph{values} $Val$. We then consider functions $V : Var \rightarrow Val$
and assume such a valuation is well-typed and can be used to evaluate an expression. Denote such a valuation by $V(e)$.
Similar to symbolic semantics, concrete semantics are given by a transition system on pairs $(s, V)$,
%Example rules are shown in \autoref{fig:trans-c}.
To relate symbolic and concrete states, we observe that a valuation can be ``composed'' with a substitution
to obtain a new valuation. Denote such a composition by $V \circ \sigma$ and note that $(V \circ \sigma)(e) = V(e\sigma)$
for all expressions $e$.
This allows us to define and prove notions of soundness and completeness, where
$\rightarrow_{S/C}^{*}$ denotes the reflexive and transitive closure of the transition relations.
\begin{theorem}[Soundness~{\cite[Thm. 2.3]{boer2021}}]
  If $(s, id, true) \rightarrow_{S}^{*} (s', \sigma, \phi)$ and $V(\phi)$ = true,
  then $(s, V) \rightarrow_{C}^{*} (s', V \circ \sigma)$
\end{theorem}
Intuitively, every symbolic execution whose path condition is true in the initial valuation $V$
corresponds to a concrete execution with the same initial valuation and the final valuation is
the result of composing with the symbolic substitution.
\begin{theorem}[Completeness~{\cite[Thm. 2.4]{boer2021}}]
  If $(s, V) \rightarrow_{C}^{*} (s', V')$ then there exist $\sigma, \phi$ such that
  $(s, id, true) \rightarrow_{S}^{*} (s', \sigma, \phi)$, $V' = V \circ \sigma$ and $V(\phi)$ = true.
\end{theorem}
Intuitively, for every concrete execution there exists a symbolic execution whose path condition
is satisfied by the initial valuation and whose final substitution recovers the final concrete valuation.
Boer et al. then extend the language with (potentially recursive) procedure calls~\cite[sec. 3]{boer2021}.
This is done by distinguishing local and global state and operating with a stack of \emph{closures} consisting
of a program fragment and the local state.
\subparagraph{Coq Implementation}
We implement this basic setup in Coq. Expressions are divided into boolean and arithmetic expressions,
and both expressions and statements are inductive types. Boolean expressions may contain arithmetic as the
arguments to a less than or equal-operator.
Valuations and substitutions are functions from strings to natural numbers and expressions respectively and
expressions are evaluated by standard functions.
We implement the transition relations as inductive predicates on pairs of configurations and take their
reflexive and transitive closure by steps to the right.
Then soundness and completeness are proved by induction on this closure and a case analysis of the final step.
The extension to procedure calls distinguishes local and global variables by construction, but is
otherwise straightforward.
\paragraph{Trace semantics}
In this section we extend the work of de Boer et al. to trace semantics in a language with a parallel operator.
Let symbolic and concrete traces be inductively defined by
\begin{align*}
  \tau_{S} ::= [\;] \mid \tau_{S} :: (x := e) \mid \tau_{S} :: b
  && \mbox{ and } &&
  \tau_{C} ::= [\;] \mid \tau_{C} :: (x := v)
\end{align*}
where $b$ is a boolean expression and $v$ a concrete value.
The final (symbolic) state of a (symbolic) trace can be recovered from an initial state by folding over
the trace and updating the state. Denote the result by $\tau\Downarrow$. Note that this is \emph{substitution}
in the case of symbolic traces, but a \emph{valuation} in the case of concrete traces.
The path condition of symbolic traces can also be recovered as the conjunction of all its boolean expressions.
Denote this path condition by $pc(\tau)$.
The semantics are given by a transition system on pairs $(s, \tau)$ -- with $\tau$ symbolic or concrete --
parametrized by an initial state. Let $\rightarrow_{\sigma}$ denote a symbolic transition from initial state $\sigma$,
and $\rightarrow_{V}$ a concrete transition from initial state $V$.
We also recover soundness and completeness in the style of de Boer et al.
\begin{theorem}[Trace Soundness]
  If $(s, [\;]) \rightarrow_{id}^{*} (s', \tau)$ and $V_{0}(pc(\tau))$ = true,\\
  then there exists $\tau'$ s.t $(s, [\;]) \rightarrow_{V_0}^{*} (s', \tau')$ and
    $V_{0} \circ (\tau\Downarrow) = \tau'\Downarrow$
\end{theorem}

\begin{theorem}[Trace Completeness]
  If $(s, [\;]) \rightarrow_{V_{0}}^{*} (s', \tau)$ there exist $\tau'$ s.t \\
  $(s, [\;]) \rightarrow_{id}^{*} (s', \tau')$, $V_{0} \circ (\tau' \Downarrow) = \tau\Downarrow$,
  and $V_{0}(pc(\tau'))$ = true
\end{theorem}
Both of these theorems are proven for the language extended with procedure calls or parallel composition
by induction on the transition relation~\footnote{In fact they both hold for appropriate initial traces (not just [])
but this formulation provides more clutter than insight}.
\paragraph{Reduction-in-Context Semantics}
As an alternative to de Boer et al.'s transition system semantics we implement reduction-in-context semantics in the
style of Felleisen~\cite{FELLEISEN1992235}.
With this approach we consider configurations of the form $(C[s], \tau)$ where $C[s]$ represent the program fragment
$s$ in the context $C$. This allows us to define a generic machinery for reductions in context and a smaller
``head reduction'' relation on $(s, \tau)$ specific to the symbolic or concrete semantics at hand.
The Coq implementation is inspired by the approach of Xavier Leroy's course on mechanized semantics~\cite{Leroy2020}.
Trace soundness and completeness can be stated as before, and are proven by induction on the reduction relation.

\paragraph{Discussion and Future Work}
We have provided an overview of de Boer and Bonsangue's formalization of SE in Coq.
Additionally we extended the formalization to include parallel composition and trace semantics,
and consider an alternative implementation of the semantics which results in more generic machinery
that can be re-used in the symbolic and concrete case.
In the future we aim to use this mechanized framework to investigate reduction techniques for symbolic
execution in concurrent settings and define useful notions of sound- and completeness for such reductions.

\bibliographystyle{eptcs}
% the .bib-file from phd proposal
\bibliography{sources}
\end{document}
